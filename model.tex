Nous avions donc besoin d'un modèle solidworks pour nous donner un aperçu de ce que donnerait le prototype une fois terminé. Aussi, nous avions besoin de ce modèle pour savoir quel support nous aurions utilisé pour installer le montage du canon Gauss, dont le transformateur qui prend énormément de place et de poids.

Le plus dur fut de créer la bobine le plus fidèlement possible. Nous avions entendu parler d'une fonction pour créer des ressorts : après maintes recherches et avec l'aide de M.Boulouch, nous avons réussi à créer un ressort à partir d'un cercle suivant une trajectoire grâce à la fonction \emph{extrusion}. La bobine ainsi créée, nous avons pu aussi choisir le nombre de révolutions et modifier sa taille pour qu'elle s'enroule autour du canon. Il faut faire bien attention à ce que le cercle soit perpendiculaire à la trajectoire sinon (si il est parallèle) la bobine sera plate et sera donc en 2D. 

Après, le premier problème fut le support. En effet, on a dû attendre d'avoir le transformateur pour savoir quelle taille il pourrait il y avoir entre les deux parties qui composent le support. Nous avons pensé à une base hexagonale qui irait bien à notre support (ainsi nous pouvons aussi centrer le transformateur et le canon plus facilement) puis lorsque nous avons récupéré le transformateur nous avons pu faire les quatre piliers.

Le deuxième problème fut l'axe supportant le canon: après quatre modifications nous avons finalement opté pour ajouter un servomoteur directement sur le canon. Nous avions aussi pensé à ajouter un trépied sur le montage, idée immédiatement supprimé car les tables du lycée sont un très bon support pour notre projet.

Ce solidworks nous permet d'avoir une première idée de ce projet avant sa réalisation, nous l'avons par la suite amélioré en agrandissant le support, les piliers et en rajoutant les équerres pour que les piliers tiennent et pour pouvoir placer la batterie. 
