\subsection{Introduction.}

Dans le cadre des TPE de Première SI, nous avons réalisé un robot prénommé Projecthor : c'est en fait un lanceur de billes automatique utilisant un système de propulsion électromagnétique appelé canon Gauss. L'objectif initial était de faire de ce système un adversaire capable de tirer sur une cible (métallique afin que les billes, magnétiques, y restent accrochées) selon  un niveau de difficulté variable, comme un robot qui pourrait vous affronter aux fléchettes. Le joueur gère sa partie ainsi que les actions du robot à partir d'un smartphone Android, connecté en bluetooth à la carte Arduino qui contrôle le système.

\subsection{Rôle.}

Je me suis occupé de la programmation de l'application Android. J'ai d'abord commencé par coder l'interface graphique en XML, afin d'avoir dès le début une idée précise de l'expérience utilisateur. Il m'a fallu consulter à de nombreuses reprises la documentation Android afin d'apprendre à utiliser les éléments d'interfaces (appelés « widgets ») les plus adaptés aux usages de l'application, par exemple une liste déroulante pour le choix du niveau de difficulté, ou ce qu'on appelle un « NumberPicker » qui permet d'entrer un nombre entier correspondant au score du joueur. 
J'ai dû ensuite collaborer avec Luc pour établir le protocole d'échange Bluetooth, par exemple avec le nom des commandes que j'envoie à la carte Arduino ('c' pour calculer l'angle de tir et l'appliquer au moteur, 'f' pour tirer, etc). J'avais déjà une expérience dans la réalisation de communications Bluetooth grâce au projet de SI en seconde, qui m'a permis d'effectuer cette partie un peu plus rapidement. Néanmoins l'année dernière l'échange ne se faisait que dans un seul sens : le téléphone envoyait les ordres au robot qui les exécutait. Cette année j'ai aussi dû lire des informations provenant de la carte Arduino, qui m'indique le score qu'elle a fait une fois le tir effectué. Lorsque Luc n'avait pas encore finalisé son programme Arduino, je réalisais les tests de communication Bluetooth avec un petit serveur réalisé manuellement en Python à l'aide de la bibliothèque PyBluez.
Une fois l'application terminée, le projet n'était lui pas encore prêt pour utiliser celle-ci ; j'ai donc aider au montage du support pour le circuit électrique, ainsi qu'à l'installation de ce dernier sur le support.

\subsection{Apports personnels.}

Ce projet m'a permis d'appronfondir mes connaissances en programmation Android, qui n'étaient au départ qu'assez superficielles. Mais j'ai surtout appris vers la fin de la réalisation, particuièrement en électronique où j'ai pu appliquer ce que l'on avait étudié et ainsi mieux le comprendre et l'intégrer, compte tenu des difficultés que j'avais eu au début de l'année en la matière. Je me suis aussi perfectionné dans la soudure, que nous avons beaucoup utilisée pour rendre les circuits plus clairs et plus robustes.
