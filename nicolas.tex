\subsection{Rôle.}

J’étais responsable avec Lucas de la fabrication du système, et en particulier du canon Gauss.

Nous avons fait les recherches du meilleur schéma de canon possible, puis nous l’avons adapté à nos besoins. Enfin nous avons monté le système et fait des tests pour vérifier son bon fonctionnement.

\subsection{Travail.}

J’ai cherché puis analysé le schéma électrique 

J’ai cherché les composants requis sur des sites spécialisés comme Farnell ou Go Tronic, puis j’ai participé à la fabrication du canon et réglé les différents problèmes qui se sont posés lors de son utilisation (et ils furent nombreux).

Ainsi, les relais de commande de charge grillaient régulièrement lors des tests de tir. Nous en avons déduit que ces relais commandés en 6 V par la carte Arduino n’étaient pas assez résistant aux forts courants appliqués dans la partie puissance. Nous avons donc décidé d’utiliser des relais 9 V, plus solides. Pour les faire commander par la carte Arduino, qui ne peut fournir que du 6V, nous avons dons dû élaborer un double système de relais permettant à la carte de commander ces relais 9 V.

Il s’agit là d’un exemple, nous avons dû résoudre de nombreux problèmes de ce type, presque toujours liés aux forts courants employés.

Nous avons également dû nous pencher sur le problème du pointage du canon. En effet, nous devions choisir entre un servomoteur, un moteur à courant continu simple ou un moteur pas à pas. Après avoir fait fausse route en décidant d’utiliser un moteur pas à pas, nous avons utilisé un servomoteur plus adapté.

Enfin, J’ai également participé au montage du canon avec tout le reste de l’équipe.


\subsection{Conclusion.}

Ce TPE m’a permis d’apprendre le travail en équipe, ainsi que la méthodologie indispensable à tout projet d’ingénierie. De plus cela m’a permis d’apprendre de nombreuses informations notamment au sujet du canon gauss,  de l’électronique et du magnétisme.











