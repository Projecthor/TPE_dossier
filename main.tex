\documentclass{report}

\usepackage[utf8]{inputenc}
\usepackage[T1]{fontenc}
\usepackage[french]{babel}
\usepackage[top=2cm,bottom=2cm,left=3cm,right=3cm]{geometry}
\usepackage{hyperref}
\usepackage{xcolor}
\usepackage{listings}
\usepackage{fancybox}
\usepackage{tikz}
\usepackage{pgfplots}
\usepackage{tocloft}
\usepackage[raccourcis]{FAST}

% Paramètres language
\lstset{basicstyle=\normalsize,
	numbers=left,
	numberstyle=\color{black}\normalsize\textbf,
	numbersep=7pt,
	backgroundcolor=\color{white},
	commentstyle=\color{black}\textit,
	keywordstyle=\color{black}\textbf,
	rulecolor=\color{black},
	stringstyle=\color{black}\textit,
	identifierstyle=\color{black},
	frame=none,
	showstringspaces=true,
}

\hypersetup{
	colorlinks=true
}

\setcounter{tocdepth}{1} % Profondeur de la table de matières

\title{Projecthor\\-- TPE --}
\author{Luc Chabassier \and Arthur Sirech \and Pablo Donato \and Nicolas Rasmont}

\begin{document}
\maketitle

\tableofcontents

\part{Introduction.}
\chapter{Contexte.}
\input intro.tex

\chapter{Règles du jeu.}
\input regles.tex

\part{Construction.}
\chapter{Modélisation.}
% \input model.tex

\chapter{Choix du matériel.} % diagramme fast
% \input mater.tex
% \input fast.tex

\chapter{Montage.}
% \input mont.tex

\part{Le canon gauss.}
\chapter{Théorie.}
% \input schem.tex

\chapter{La réalisation.}
% \input rea.tex

\part{Aspect logiciel.}
\chapter{L'application android.}
% \input andro.tex

\chapter{Le programme arduino.}
% \input ardui.tex

\part{Annexes.}
\appendix
% \makeatletter
% \def\@seccntformat#1{Annexe~\csname the#1\endcsname:\quad}
% \makeatother

\chapter{Calculs de balistique.}
% \input balis.tex

\chapter{Fiches personnelles.}

\newpage
\listoffigures
\addcontentsline{toc}{chapter}{Liste des figures.}

\end{document}

