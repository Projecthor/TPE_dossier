Le fonctionnement est divisé en deux phases, le chargement (\ref{gauss_charg}) et le
tir (\ref{gauss_tir}).

\section{Chargement.} \label{gauss_charg}

Lors du chargement le circuit de relais est activé par la
carte Arduino selon une fréquence de 10Hz. Ceci crée un
signal périodique (donc des variations de courant) qui permet
au transformateur de transmettre au circuit secondaire un
courant électrique d'une tension plus forte mais d'une
intensité plus faible.

Ce courant va permettre de charger les condensateurs.
Cependant, si nos condensateurs ont une tension aux bornes
maximales de $200 V$, il n'atteindrons jamais cette limite. En
effet la tension aux bornes en fonction du temps de charge $T$ en
secondes suit une croissance logarithmique: au bout d'une
certaine durée la tension augmente très lentement puis stagne.
On a atteint l'asymptote de la courbe.

À la suite d'une campagne de test cette durée a été fixée à $30
s$, au bout de laquelle la tension stagne à $150 V$.

Pour éviter que le courant passe de la borne négative à la borne
positive des condensateurs via le transformateur sitôt le
chargement terminé, on utilise une diode THT qui bloque le
passage du courant dans ce sens.

Après le temps de chargement convenu, la carte arduino
arrête le circuit de relais: les condensateurs sont
chargés. L'énergie stockée par les condensateurs est donnée par
la formule:
\begin{equation}
	E = \frac{1}{2}Cu^2
\end{equation}

Avec $E$ en Joules, $C$ la capacité en Farads et $u$ la tension aux
bornes en   Volts. Cette formule va nous permettre de calculer
l'efficacité de notre   canon (nous y reviendrons plus tard).

\section{Tir.} \label{gauss_tir}

Lorsque la commande de tir est envoyée, le circuits de relais
 est activée par la carte. Le tir peut également être
déclenché manuellement avec l'interrupteur.

Ce circuit ferme la maille et fait passer le courant dans la
bobine, qui transforme l'énergie électrique en champ
magnétique. La ligne de champ au milieu de la bobine est
rectiligne: c'est elle qui va propulser la bille
ferromagnétique. La puissance du champs magnétique est donnée
par la formule:
\begin{equation}
	\phi = L*I
\end{equation}

Avec $\phi$ le flux du champs magnétique en Weber, $L$ l'inductance
en Henry et $I$ l'intensité en Ampères.

Pour améliorer la vitesse de propulsion de la bille, on peut
donc soit   augmenter l'intensité du courant, soit augmenter
l'inductance de la   bobine.

Cette dernière est donnée par la formule :
\begin{equation}
	L=\frac{\mu_0*N^2*S}{l}
\end{equation}

Avec $L$ l'inductance en Henry, $\mu_0$ la constante magnétique, ($\mu_0=4*10^{-7} H·m^{-1}$ ), $N$ le nombre de
spires, $S$ la section de la bobine en $m^2$ et $l$   la longueur de
la bobine en mètre.

Après avoir fabriqué notre première bobine, nous avons remarqué
que   la vitesse de la bille était faible et qu'il fallait donc
optimiser   l'inductance de la bobine. Pour cela nous avons
refabriqué notre bobine   avec la même longueur de fil de
cuivre mais en divisant par deux sa   longueur, ce qui a
également doublé sa section. Nous allons calculer
l'inductance de la première bobine puis de la deuxième, afin de
calculer   le gain en inductance(ce calcul reste toutefois
assez approximatif).

\subsection{Première bobine.}
La première bobine avait une longueur $l_{max}$ de $100\ mm$, un
diamètre   extérieur $d_{ext}$ de $15\ mm$ et un diamètre intérieur
$d_{int}$ de $8\ mm$,   utilisait une longueur $l_{cu}$ de $20000\ mm$ de fil
de cuivre émaillé de   diamètre $d_{cu}=0,5\ mm$.

On calcule la longueur de fil moyenne $l_{moy}$ contenue dans une
spire: \\
\begin{math}
	l_{moy}= \pi*(d_{int}+\frac{d_{ext}-d_{int}}{2}) \\
	l_{moy}=\pi*(8+\frac{15-8}{2})=36\ mm
\end{math}

On calcule le nombre de spire N de la bobine: \\
\begin{math}
	N=l_{cu}/l_{moy} \\
	N= 20000/36 = 555\ spires
\end{math}

On calcule la section S de la bobine: \\
\begin{math}
	S= \pi*[(\frac{d_{ext}}{2})^2 - (\frac{d_{int}}{2})^2] \\
	S= \pi*[(\frac{15}{2})^2 - (\frac{8}{2})^2] = 126\ mm^2 = 0,0126 m^2
\end{math}

On calcule l'inductance $L_{1}$ de la bobine: \\
\begin{math}
	L_{1} = \frac{\mu_0.N^2.S}{l} \\
	L_{1} = \frac{4\pi*10^{-7}*555^2*0.0126}{0.100}=0.049 H
\end{math}

\subsection{Bobine utilisée.}
La deuxième bobine avait une longueur $l_{max}$ de $50\ mm$, un
diamètre   extérieur $d_{ext}$ de $30\ mm$ et un diamètre intérieur
$d_{int}$de $8\ mm$,   utilisait une longueur $l_{cu}$ $20000\ mm$ de fil
de cuivre émaillé de   diamètre $d_{cu}$ $0,5\ mm$.

On calcule la longueur de fil moyenne $l_{moy}$ contenue dans une
spire: \\
\begin{math}
	l{moy}= \pi*\frac{d_{int}+(d_{ext}-d_{int})}{2} \\
	l{moy}= \pi*\frac{8+30-8}{2}= 60\ mm
\end{math}

On calcule le nombre de spire $N$ de la bobine: \\
\begin{math}
	N = \frac{l_{cu}}{l_{moy}} \\
	N = \frac{20000}{60} = 333\ spires
\end{math}

On calcule la section $S$ de la bobine: \\
\begin{math}
	S = \pi*[(\frac{d_{ext}}{2})^2 - \frac{(d_{int}}{2})^2] \\
	S = \pi*[(\frac{30}{2})^2 -(\frac{8}{2})^2] = 657\ mm^2 = 0,0657\ m^2
\end{math}

On calcule l'inductance $L_{2}$ de la bobine: \\
\begin{math}
	L_{2} = \frac{\mu_0*N^2*S}{l} \\
	L_{2} = \frac{4\pi*10^{-7}*333^2*0.0657}{0.05}= 0.18\ H
\end{math}

\subsection{Conclusion.}
Le gain en inductance $G$ (donc en efficacité) est de: \\
\begin{math}
	G = 100*L_{2}/L_{1} \\
	G = 100 * \frac{0.18}{0.049} = 367\%
\end{math}

Il y a donc un gain de $367\%$ d'efficacité.

