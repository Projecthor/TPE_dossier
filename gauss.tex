Un canon Gauss (ou coilgun, de coil, bobine en anglais) est un
système utilisant le magnétisme pour propulser un projectile.
Il fait partie de la famille des canons électromagnétiques, qui
comprend également le canon électrique (ou railgun) qui exploite
la force de Laplace pour envoyer un projectile.

La marine américaine s'est notamment intéressée au concept de
canon électrique pour l'installer sur ses navires. En effet,
pour envoyer un obus à une vitesse 20\% supérieure, il faut
doubler la masse de poudre initiale. Les canons
électromagnétiques sont donc utiles pour envoyer des
projectiles à des vitesses très importantes, inatteignables par
d'autres moyens.

Bien entendu il n'est pas question pour nous d'accélérer nos
billes magnétiques à Mach 12! Mais le canon Gauss, de par les
principes physiques qu'il met en œuvre, était un sujet d'étude
intéressant. C'est pourquoi, dans cette partie, nous allons
détailler les principes de fonctionnement de notre canon,
ainsi que sa conception.

Il existe de multiples variantes de canons Gauss, qui diffèrent
essentiellement par leurs systèmes de chargement plus ou moins 
efficaces et complexes. Cependant, tous ces canons utilisent
des composants communs qui sont essentiels au canon:
\begin{itemize}
\item La ou les bobines, qui servent à transformer l'énergie
électrique en champ magnétique. Ce dernier servira à propulser
le projectile ferromagnétique. La bobine varie d'un canon à
l'autre par sa longueur, son nombre de spires, son diamètre...
Toutes ces caractéristiques conditionnent l'inductance de la
bobine (nous en reparlerons plus loin)

\item Les condensateurs, qui servent à accumuler l'énergie électrique
nécessaire à la production du champ magnétique. Les canons
Gauss utilisent des condensateurs d'une capacité électrique et
d'une tension aux bornes variables. Ses caractéristiques vont
influer sur l'énergie électrique transmise à la bobine.

\item L'alimentation, qui permet le chargement des condensateurs. Elle
peut être périodique ou continue et d'une tension et intensité
variables. C'est elle qui va déterminer le temps de chargement
des condensateurs.
\end{itemize}

