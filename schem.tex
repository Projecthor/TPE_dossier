Aux termes de multiples recherches nous sommes arrivés au
schéma de canon Gauss de la figure~\ref{oldcircuit}, qui avait l'avantage d'être assez
simple et d'utiliser des composants usuels.

\input oldcircuit.tex

Néanmoins, nous avons modifié ce montage pour qu'il corresponde
mieux à nos besoins.

Toute la partie gauche du schéma électrique est une partie
\emph{commande}; c'est elle qui, en ouvrant et fermant le
relais à une fréquence de l'ordre d'une dizaine de Hertz, va
créer le signal périodique nécessaire au bon fonctionnement du
transformateur. Nous l'avons supprimée, cette fonction étant
assurée par la carte Arduino: la carte, à l'aide d'un
programme informatique, active le relais selon une fréquence
donnée.

La lampe néon ainsi que sa résistance de $300\ k\Omega$ dont la
fonction est de signaler la fin du chargement n'est pas utile
car la carte arduino arrête automatiquement le chargement au
bout d'un temps que nous avons déterminé.

L'interrupteur de mise à feu est dans notre montage doublé par
un relais contrôlé par la carte qui active le tir sur l'ordre
de l'application Bluetooth. Nous avons néanmoins gardé
l'interrupteur en cas de dysfonctionnement du relais.

Lors des tests nous avons constatés que les deux relais
contrôlés par la carte Arduino grillaient à l'usage, car les
forts courants que nous leurs appliquions dans la partie
\emph{puissance} créaient des arcs électriques les endommageant
irrémédiablement. Nous avons donc du utiliser un double montage
de relais: les relais 5V de la carte Arduino activent des
relais 9 V plus résistants aux forts courants.

Au final le montage que nous avons réalisé est représenté par la figure \ref{circuit}.

\input circuit.tex

