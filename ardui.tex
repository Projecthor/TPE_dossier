% Chapitre sur le programme arduino

\section{Présentation.}
Bien qu'il soit habituellement considéré que l'arduino est un language dérivé du \emph{C}, c'est en fait du \emph{C++}. Par conséquent, le programme arduino de ce robot est codé en C++.

Les outils utilisés sont :
\begin{description}
	\item[git] Est utilisé pour la gestion des versions et l'enregistrement des modifications du programme. Le code est hébergé sur le site \emph{github} à l'adresse \url{https://github.com/Projecthor/arduino}.
	\item[make] Est utilisé pour automatiser les taches de compilation et transfert du programme.
	\item[arduino] Ce n'est pas l'environnement de développement arduino qui est utilisé mais juste les bibliothèques C++ fournies.
	\item[avr-gcc] Une carte arduino se basant sur une puce AVR, c'est le compilateur utilisé pour générer le binaire.
	\item[avr-tools] Est utilisé pour le transfert vers la carte et différentes opérations sur le binaire.
	\item[vim] Pour écrire le code.
	\item[screen] Pour ouvrir un terminal serial avec l'arduino.
\end{description}

\section{Objectif}
Ce programme doit contrôler le robot. C'est à dire qu'il doit :\begin{itemize}
	\item mesurer la distance à la cible.
	\item détecter le obstacles.
	\item calculer l'angle nécessaire.
	\item gérer les niveaux de difficulté.
	\item recevoir et interpréter les ordres du téléphone.
	\item charger les condensateurs.
	\item déclencher le tir.
\end{itemize}

Les niveaux de difficulté sont détaillés au chapitre~\ref{regles}, page~\pageref{regles} et les calculs de balistiques en annexe~\ref{balis}, page~\pageref{balis}.

\subsection{Protocole bluetooth.} \label{prot_bt}
Le protocole bluetooth se déroule en un certain nombre d'étapes :\begin{enumerate} % TODO logigramme
	\item Le smartphone envoie le niveau de difficulté à la carte arduino.
	\item Le smartphone envoie l'ordre de charger les condensateurs.
	\item La carte android indique au smartphone la fin du chargement.
	\item Le smartphone envoie l'ordre de tirer.
	\item La carte android envoie son score.
	\item On retourne à l'opération 2.
\end{enumerate}

Ce protocole est suivit jusqu'à ce que le smartphone envoie à la carte arduino l'ordre de fin.

\section{Fonctionnement.}
Comme c'est un code C++, le programme est structuré en classes. Chacune a un rôle précis et elles sont toutes utilisées dans la fonction \emph{main} (voir section~\ref{ard_main}).

Les classes sont les classes \emph{Bluetooth}~(\ref{ard_bt}), \emph{Canon}~(\ref{ard_can}), \emph{Motor}~(\ref{ard_mot}) et \emph{Game}~(\ref{ard_game}).

\subsection{Classe Bluetooth.} \label{ard_bt}
La classe Bluetooth, qui gère la connection bluetooth, est définie comme suit :
\begin{lstlisting}[language=C++]
class Bluetooth
{
	bool waitForConnection(callback cb, void* data);
	unsigned int receive(char* buffer, unsigned int size);
	template<typename T> bool send(T buffer);

	// Membres internes
};
\end{lstlisting}

Les noms des fonctions membres sont assez explicites :\begin{description}
	\item[waitForConnection] Attend une connection entrante du smartphone.
	\item[receive] Reçois des valeurs sous leur forme binaire.
	\item[send] Envoie des valeurs de n'importe quel type grâce au fonctionnement des templates.
\end{description}

La classe utilise en interne une interface serial avec le shield bluetooth.

\subsection{Classe Canon.} \label{ard_can}
La classe Canon, qui gère le chargement des condensateurs et le déclenchement du tir, est définie comme suit :
\begin{lstlisting}[language=C++]
const unsigned long loadTime = 30000;

class Canon
{
	void load();
	void fire();

	// Membres internes
};
\end{lstlisting}

La constante \emph{loadTime} définit le temps de chargements des condensateurs en millisecondes.

Les fonctions membres ont là aussi des noms assez explicite :\begin{description}
	\item[load] Charge les condensateurs en ouvrant et fermant un relai à une fréquence données, pendant le temps définit par \emph{loadTime}.
	\item[fire] Déclenche le tir en fermant le relai reliant les condensateurs à la bobine.
\end{description}

\subsection{Classe Motor.} \label{ard_mot}
La classe Motor gère juste l'orientation du canon :
\begin{lstlisting}[language=C++]
class Motor
{
	int currentAngle() const;
	int destAngle() const;

	Motor& operator=(int angle);
	Motor& operator+=(int angle);
	Motor& operator-=(int angle);
	void update();

	// Membres internes
};
\end{lstlisting}

Ici, les fonctions membres sont un peu moins explicites :\begin{description}
	\item[currentAngle] Indique l'orientation actuelle du moteur.
	\item[destAngle] Indique l'orientation qu'aura le moteur après un appel à \emph{update}.
	\item[operator=,operator+=,operator-=] Change l'orientation du canon en interne.
	\item[update] Incline le canon en fonction de l'orientation interne.
\end{description}

Cette classe Motor se contente d'utiliser la classe \emph{Servo} d'arduino en interne. Sont véritable intérêt est de permettre de changer le moyen d'orienter le canon sans modifier le reste du code.

\subsection{Classe Game.} \label{ard_game}
La classe game est un peu plus complexe, car elle a plusieurs rôles :
\begin{lstlisting}[language=C++]
class Game
{
	bool waitForDifficulty();
	void getDistance();
	bool waitComputeOrder();
	bool waitFireOrder();
	int computeAngle();
	int getScore();
	bool checkSecurity();

	// Membres internes
	// Constantes de balistiques
};
\end{lstlisting}

Les constantes de balistiques (voir chapitre~\ref{balis}, page~\pageref{balis}) sont entrées dans les membres internes de cette classe.

Les différents rôles de cette classe sont l'usage du capteur de distance, le calcul de l'angle et de la balistique, le test de sécurité, et la recéption des ordres bluetooth. Pour cette dernière fonction, cette classe utilise la classe Bluetooth (\ref{ard_bt}).

Les différents membres servent à interfacer ces rôles pour la fonction main : \begin{description}
	\item[waitForDifficulty] Attend le niveau de difficulté par le smartphone. Retourne \emph{false} en cas d'erreur à la réception.
	\item[getDistance] Capte la distance et la stocke en mémoire.
	\item[waitComputeOrder] Attend l'ordre de charger les condensateurs.
	\item[waitFireOrder] Attend l'ordre de déclencher le tir.
	\item[computeAngle] Choisit la position du tir sur la cible en fonction du niveau de difficulté et calcule l'angle nécessaire.
	\item[getScore] Retourne le score correspondant au dernier appel à \emph{computeAngle}.
	\item[checkSecurity] Vérifie s'il y a un obstacle sur la trajectoire du tir.
\end{description}

\subsection{La fonction \emph{main}.} \label{ard_main}
Cette fonction est primordiale car elle agence tous les autres éléments du code pour faire le programme en lui-même.

Nous ne détaillerons pas ici le code en lui-même mais l'algorithme utilisé.

\begin{algorithm}[H]
	\SetAlgoLined
	\KwData{connected = faux}
	\While{forever}{
		\eIf{connected est faux}{
			On reçoit le niveau de difficulté.\\
			On mesure la distance à la cible.\\
			On met \emph{connected} à vrai.
		}{
			\If{erreur à la réception de l'ordre de chargement}{
				On met \emph{connected} à faux.\\
				On retourne au début de la boucle.
			}
			On charge le canon.\\
			On calcule l'angle.\\
			On oriente le canon.\\
			\If{erreur à la réception de l'ordre de tir}{
				On met le moteur en diagonale.\\
				On met \emph{connected} à faux.\\
				On retourne au début de la boucle.
			}
			On déclenche le tir.\\
			On envoie le score au smartphone.\\
			On met le moteur en diagonale pour faciliter le chargement.
		}
	}
	\caption{Fonction \emph{main}.}
\end{algorithm}




