\subsection{Introduction.}
En tant que projet des travaux personnels encadrés, nous avons en groupe choisi de réaliser un lanceur de bille automatique contrôlé par Bluetooth depuis une application Android et fonctionnant grâce au système du canon Gauss. Au sein de ce projet, je me suis occupé en collaboration avec Nicolas de la partie mécanique, ce qui inclut la recherche des solutions techniques, la commande des composants, les tests ainsi que la construction matérielle de notre TPE. Il s'agit d'une partie importante qui nécessite l'entraide d'au moins deux membres de notre groupe afin d'obtenir un résultat le plus fonctionnel possible. Bien entendu, un travail en parallèle est souvent opéré pour que notre productivité soit optimale.

\subsection{Travail.}
Premièrement, il a fallu établir la liste des solutions répondants au fonctions  techniques du lanceur de bille. Nous avons du déterminer la solution la plus simple et efficace qui permette de faire pivoter le canon pour que sa trajectoire varie et que l'on puisse viser avec plus ou moins de précision une cible située à une distance variable. Ayant écarté la solution d'un moteur à courant continu simple car trop imprécis, nous nous somme tournés vers un moteur pas à pas qui, en augmentant sa précision à l'aide d'un réducteur, est satisfaisant. Cependant, après quelques séances, nous avons découvert qu'un servomoteur était le type de moteur qui nous fallait: en effet, celui-ci est bien plus simple au niveau de la programmation, s'ajustant lui-même au moyen d'un potentiomètre il ne nécessite pas de facteur de réduction et enfin il possède un couple largement satisfaisant ($3\ kg.cm^{-1}$) pour soulever notre canon (de l'ordre de $200g$). Pour la propulsion de la bille nous avons pensé à un système à ressort, ou à un système pneumatique mais le canon Gauss s'est révélé être précis, plus puissant que le ressort et, au contraire du système pneumatique, réalisable car nécessitant des composants communs. Nous avons regroupé toutes ces solutions dans un diagramme fast afin qu'elles soient lisibles aisément.

Nous nous sommes concentré par la suite sur le canon Gauss et après diverses recherches nous avons trouvé un schéma que nous avons pu adapter en fonction de nos besoins et du matériel à notre disposition: par exemple, disposant d'une carte Arduino programmable nous avons pu remplacer une partie par un programme et certains interrupteurs par des relais. Après acquisition des composants, nous nous sommes lancés dans le montage du circuit. Il s'agit, si on simplifie, de charger deux condensateurs qui délivreront au moment voulu leur puissance dans une bobine qui génère donc un champ magnétique propulsant la bille ferromagnétique. Enfin en groupe nous avons réaliser le support en se basant sur le modèle réalisé par Arthur pour obtenir ce que nous avons aujourd'hui.

\subsection{Problèmes.}
Nous avons rencontré quelques problèmes, notamment lors du montage du circuit du canon Gauss. En effet, il a tout d'abord fallu un temps de compréhension afin de savoir parfaitement à quoi servait le circuit. Nous avons, pour optimiser notre bobine qui ne tirait pas loin, calculer son inductance et avec la formule de celle-ci nous avons pu en reconstruire une d'une inductance environ $367\ \%$ supérieure selon nos calculs. A la suite de plusieurs tests qui faisait griller nos relais $5V$ d'Arduino, nous avons fait des montages de double relais: les relais $5V$ d'Arduino activent désormais des relais plus résistants de $9V$ alimentés par une pile adaptée.

\subsection{Conclusion.}
La réalisation de ce projet de groupe à été une expérience très enrichissante qui représente bien le métier d'ingénieur qui doit procéder par étapes et travailler en équipe. C'est une approche technique et scientifique du monde professionnel de l'ingénierie qui encourage l'investissement personnel de chacun dans le projet. De plus, c'est très satisfaisant de réaliser et de voir fonctionner son propre projet.

