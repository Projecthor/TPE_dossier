Au départ, on était pas du tout parti sur une idée de canon, de fléchette, de cible, etc... On était parti sur un moteur Sterling, au final trop dur à réaliser et trop couteux et ne montrant pas assez de principes de physique.

On voulait quelque chose qui pourrait plaire au Jury tout en intégrant des principes de physiques un peu complexe ainsi que des montages électriques incluant plus de composant qu'on avait appris durant le début de l'année, on voulait montrer qu'on pouvait produire quelque chose d'ingénieux en travaillant comme une vrai équipe.

On a eu l'idée du canon par hasard, une personne de notre groupe de 5 étant féru de tout ce qui touche à l'armement nous a parlé d'un canon pouvant propulser des objets métalliques sans poudre ou quelconque explosion mais juste en se servant du champ magnétique : le Canon Gauss.

Après plusieurs recherches sur Internet nous nous sommes mis d'accord pour faire un jeu de fléchette où l'on pourrait jouer tout seul contre la machine selon certains niveaux de difficultés. On a du tout d'abord revoir le problème du projectile, les fléchettes étant trop lourdes et ayant des parties non-métallique, impossible pour un petit canon de la taille d'un simple stylo de propulser autant de poids. C'est alors que m'est venu l'idée de me servir de billes magnétiques pour remplacer les fléchettes : elles sont moins lourdes, plus compact, entierrement métallique, disponible en grande quantité pour les test et surtout elles font parfaitement la taille du canon !

Après cela, j'ai été en charge de la partie modélisation grâce au logiciel Solidworks pour permettre à mon équipe de pouvoir se donner une idée du support sur lequel  le canon pourrait être disposés. 

Je suis parti sur un support hexagonal (plus facile pour placer les piliers entre les deux plateformes ainsi que pour centrer les composants), ensuite j'ai du évoluer avec le reste du projet. 

On voulait d'abord faire bouger le canon avec un axe grâce à un moteur pas à pas avec des engrenages. J'ai donc du faire le moteur moi-même et ajuster les contraintes pour que les engrenages tournent avec le moteur.

Ces engrenages et ce moteur ont vite été remplacé par un servomoteur que j'ai du modélisé à son tour.

Après avoir modélisé le support, je suis passé à la phase manuelle et j'ai donc fabriqué le support et les piliers avec du bois qu'on a récupéré. 

Après avoir finit ma phase de modélisation et ma phase manuelle, il me restait plus qu'à aider mon équipe à faire les soudures, câbler le canon, faire les tests et les aider dans les tâches qui demandait plusieurs personnes.

Les $\frac{3}{4}$ de mon temps de travail ont été d'aider mes coéquipiers, les $\frac{1}{4}$ restant sont la phase modélisation.
